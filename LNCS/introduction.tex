%As the digital transformation deepens across various sectors, the demand for more powerful, energy-efficient, and smaller semiconductors continues to escalate. This surge in demand places immense pressure on semiconductor manufacturers to scale up production without compromising quality. Semiconductor production involves hundreds of sophisticated steps, each of which must be precisely controlled to ensure the functionality and yield of the final product \cite{kopp2020smt2020}. The complexity is exacerbated by the rapid pace of innovation in semiconductor design, which frequently shifts production parameters and process requirements.

Dispatching and scheduling are critical in semiconductor manufacturing because they directly impact the throughput and utilization of resources. Dispatching refers to the process of assigning work to specific machines in real-time, while scheduling involves pre-planning the sequence and timing of operations to optimize certain objectives, such as minimizing the total completion time or balancing the load across different machines~\cite{schumann2022scheduling}.

For smaller volumes at workcenters in the semiconductor fab, which can be modeled as flexible job shops, optimal solutions can be achieved using mathematical optimization \cite{waschneck2018deep}. However, in larger and more dynamic environments, the complexity and computational time constraints limit the feasibility of applying mathematical optimization. Consequently, optimization is typically implemented on a local scale, isolated to individual workcenters. In complex scenarios, this localized approach to production scheduling may lead to solutions that are globally sub-optimal.

Dispatching involves determining which job should be performed at which time, by which resource. This decision is based on the availability of resources and the urgency of job orders \cite{Hopp2011}. However, as production scenarios become more complex, particularly in high-tech industries such as semiconductor manufacturing, the static nature of traditional dispatching rules often proves insufficient. These rules typically do not account for the rapid changes in production conditions or the intricate dependencies between different production processes.  
Traditional dispatching solutions often fall short in such an environment, as they cannot adequately adapt to the dynamic and complex nature of semiconductor production. This inadequacy can lead to sub-optimal decisions that compromise operational efficiency \cite{Uzsoy1992}.

The complexity of scheduling in large-scale semiconductor production, a cornerstone of operational management, is significantly influenced by the unpredictable nature of processing times and machine availabilities. Variability in processing times can arise from differences in equipment performance, material handling durations, or the unique characteristics of each set of wafers, while frequent machine breakdowns can cause substantial disruptions, underscoring the need for strong and adaptable scheduling strategies \cite{leachman1996benchmarking}. Scheduling involves meticulously planning and organizing production activities to ensure optimal resource utilization and synchronized product flows across various manufacturing stages, which is crucial not only for maintaining high throughput but also for minimizing makespan \cite{schumann2022scheduling}. The task is particularly daunting in semiconductor manufacturing due to the high variability in production processes, the sensitive nature of the materials, and stringent quality requirements \cite{May2006}. Each semiconductor product may pass through hundreds of processing steps, requiring precise timing and coordination. The complexity is further compounded by the need to accommodate a high mix of product types, each with distinct processing needs and priorities, and to integrate new product introductions seamlessly into the production schedule without disrupting ongoing operations \cite{Moench2011}.

In this paper, we explore the complexities of scheduling within large-scale Semiconductor Manufacturing Scheduling Problems (SMSP), commonly referred to as fabs. Fabs are intricate production settings distinguished by their specialized job flows and advanced machinery. Modern semiconductor fabs manage the processing of tens of thousands of operations daily across over 1000 different machines \cite{kopp2020smt2020}. These machines are grouped by functionality—such as diffusion, etching, or metrology—and each step in the manufacturing process can be assigned to any machine that offers the necessary capabilities. The scheduling challenge in semiconductor production thus involves assigning specific operations to appropriate machines and determining the optimal sequence of these operations on each machine. Additionally, schedules must be flexible and capable of rapid adjustments in response to machine failures or deviations in the process.

In our previous work \cite{Ali2024}, we proposed a Greedy Search based Ant Colony Optimization (GSACO)
algorithm for (re-)scheduling semiconductor production operations. 
Our algorithm harnesses Ant Colony Optimization (ACO) \cite{Dorigo2019} for exploration, while
Greedy Search (GS) \cite{Papadimitriou} enables responses in short time. 
In this way, GSACO overcomes limitations of a state-of-the-art Constraint Programming approach \cite{Perron2023}
on large-scale SMSP instances. Our approach synergistically combines probabilistic operation sequencing with a greedy machine assignment strategy, targeting up to five operations per lot with the primary objective of minimizing makespan. Building on this foundational work, this paper extends these initial concepts by proposing an enhanced approach, GSACO-O, specifically designed to optimize operational throughput in addition to minimizing makespan. This development marks a significant advancement in our methodology, aiming to refine the dispatching rules further though simulation.

The paper is organized as follows. 
Section~\ref{sec:lit_rev} provides literature review.
In Section~\ref{sec:problem_f}, we formulate SMSP in terms of the well-known flexible job shop scheduling problem (FJSSP). 
% and Section \ref{sec:aco} describes the overview of the ACO approach. 
Our GSACO-O algorithm is presented in Section~\ref{sec:gsaco}.
In Section~\ref{sec:sim} we present the customized simulation adopted from \cite{Kovacs2022}.
Section~\ref{sec:results} provides and discusses experimental results.
Finally, Section~\ref{sec:conclusion} concludes the paper.


%The complexity of scheduling is intensified by the unpredictable nature of processing times and machine availabilities. Variability in processing times may arise from differences in equipment performance, material handling durations, or the unique characteristics of each set of wafers. Furthermore, frequent machine breakdowns can lead to substantial disruptions, underscoring the need for strong and adaptable scheduling strategies \cite{leachman1996benchmarking}.

%Scheduling in large-scale semiconductor production is a cornerstone of operational management that directly influences the effectiveness and efficiency of the entire manufacturing process \cite{schumann2022scheduling}. It involves planning and organizing production activities to ensure that resources are utilized optimally and product flows are synchronized across various stages of the manufacturing process. Effective scheduling is critical not only for maintaining high throughput but also for minimizing makespan.

%The scheduling of semiconductor manufacturing is notably complex due to the high variability in production processes, the sensitive nature of the materials involved, and the stringent quality requirements \cite{May2006}. Each semiconductor product may pass through hundreds of processing steps, requiring precise timing and coordination. Moreover, the high mix of product types, each with different processing needs and priorities, adds another layer of complexity \cite{Mönch2011}. This complexity is further exacerbated by the need to integrate new product introductions seamlessly into the production schedule without disrupting ongoing operations.

%Scheduling production operations within the intricate landscape of semiconductor manufacturing represents one of the most daunting challenges in resource allocation. The dynamic nature of manufacturing environments—characterized by unpredictable fluctuations in process and demand, supply chain delays, and frequent machine breakdowns—significantly intensifies this complexity. Consequently, production scheduling must not only strive for near-optimal outcomes but also maintain robustness and flexibility to adapt swiftly to frequent changes in the production landscape. 