The structure of our enhanced GSACO-I algorithm, along with its input and internal parameters, 
is summarized in Table~\ref{tab:parameters} and 
illustrated in Figure~\ref{fig:aco-flowchart}. 
The four main submodules, highlighted in bold, are discussed in detail in the following subsections.

\begin{table}[t]
	\caption{GSACO parameters}\label{tab:parameters} \centering
	\begin{tabular}{|l|l|}
		\hline
		Parameter & Description \\ \hline
		$o$ & Objective \\
		$s$ & State \\
		$l$ & Cycles/time limit        \\
		$n$ & Operations per lot \\
		$h$ & Planning period \\
		$k$ & Number of ants \\
		$\tau_{y}$ & Initial pheromone level \\
		$\tau_{z}$ & Minimum pheromone level \\
		$\tau_{e}$ & Pheromone level on edge $e$ \\
		%	$\eta_{e}$ & Heuristic information on edge $e$ \\
		$\rho$ & Evaporation rate \\
		$c$ & Contribution of best schedules \\
		%	$\alpha$ & Influence of pheromone $\tau_{e}$ \\
		%	$\beta$ & Influence of heuristic $\eta_{e}$    \\
		\hline
	\end{tabular}
\end{table}

\begin{figure}[t]
	\includegraphics[width=\textwidth]{aco-flowchart.png}
	\caption{GSACO-I framework}
	\label{fig:aco-flowchart}
\end{figure}

Moreover, Algorithm~\ref{gsaco} provides a pseudo-code representation of
GSACO-I for minimizing makespan and optimizing operations.

For a configurable cycle number or time limit~$l$,
each of the $k$ ants applies greedy search
using the GS procedure with respect to the objective.
That is, the first GS phase constructs an operation sequence, which is
then taken as basis for greedily assigning the operations to machines
in the second phase.   
Note that the ants run independently, so that their GS trials
can be performed in parallel.
As a result, $k$ schedules along with edges between operations
(described in Subsection~\ref{subsec:initialization})
that have been selected for their construction are obtained.
If some of these schedules improves the makespan over the best
schedule found in previous iterations (if any),
the best schedule gets updated.
As common for ACO algorithms,
pheromones $\tau_e$ on edges~$e$ are subject to evaporation,
according to the formula $\rho\cdot\tau_e$,
while edges selected to construct the best schedule obtained
so far also receive a pheromone contribution,
calculated as $\tau_e+c$.
Such pheromone deposition increases the chance for edges contributing to the
current best schedule
to get re-selected % by the GS procedure
in forthcoming iterations.

\begin{algorithm}[ht]
	\caption{Greedy Search based ACO (GSACO) for Scheduling}
	\label{gsaco}
	\KwIn{instance, $l$, $o$, $s$, $n$, $h$, $\text{objective}$}
	\KwOut{best schedule found by ants}
	\KwParam{$k$, $\tau_{y}$, $\tau_{z}$, $\rho$, $c$}
	Initialize adjacency, pheromone, and machine matrix\; 
	\eIf{\text{objective} == \text{Makespan}}{
		$\mathit{makespan}\leftarrow \infty$\;
	}{
		$\mathit{operations}\leftarrow 0$\;
	}
	\While{cycle or time limit $l$ is not reached}{
		\ForEach{ant \KwFrom $1$ \KwTo $k$}{
			Run GS procedure to find a schedule\;
		}
		\eIf{\text{objective} == \text{Makespan}}{
			$\mathit{new}\leftarrow$ shortest makespan of ants' schedules\;
			\If{$\mathit{new}<\mathit{makespan}$}{
				$\mathit{makespan}\leftarrow \mathit{new}$\;
				$\mathit{best}\leftarrow$ an ant's schedule of minimum $\mathit{makespan}$\;
			}
		}{
			$\mathit{new}\leftarrow$ maximum operations of ants'\;
			\If{$\mathit{new}>\mathit{operations}$}{
				$\mathit{operations}\leftarrow \mathit{new}$\;
				$\mathit{best}\leftarrow$ an ant's schedule of maximum $\mathit{operations}$\;
			}
		}
		\ForEach{edge $e$ in pheromone matrix}{
			$\tau_{e} \leftarrow \max\{\rho\cdot\tau_e,\tau_z\}$\tcp*[r]{evaporation}
		}
		\ForEach{edge $e$ selected by $\mathit{best}$ ant}{
			$\tau_{e} \leftarrow \tau_e+c$\tcp*[r]{deposit pheromones}
		}
	}
	\Return $\mathit{best}$\;
\end{algorithm}


\subsection{Input Module}
This module reads in an SMSP instance from the SMT2020 simulator develop by \cite{Kovacs2022}, 
and SMT2020 dataset \cite{kopp2020smt2020}.
The instance includes tool groups with their respective machines, jobs currently in progress, and production routes. In terms of dynamic state, the processing times for these routes are stochastic. Conversely, the simulator provides observed averages for processing times and job releases. Additionally, the module takes several inputs for the GSACO optimization process: the objective~$o$, the state of the fab~$s$, planning period~$h$, 
operation per lot~$n$ and a limit~$l$ on either the number of cycles or the time allocated for optimization.


\subsection{Initialization Module}
\label{subsec:initialization}
In view of long production routes with hundreds of operations
in the SMT2020 dataset, we introduce a configurable planning horizon~$n$
as upper bound on the length $n_j$ of the operation sequence for a job~$j$.
The planning horizon thus constitutes a scaling factor for the size and
the resulting complexity of SMSP instances.
% This horizon is crucial in planning and decision-making, especially within large-scale production systems. 
In practice, unpredictable stochastic events make long-term schedules obsolete and necessitate frequent re-scheduling,
where limiting the planning horizon upfront provides a means to
control the search and enable short response times.

To express SMSP as a search problem on graphs,
we identify an instance with the disjunctive graph

whose vertices~$V$ contain the operations $O_{i,j,t}$ plus
a dummy start node~$0$,
conjunctive edges\linebreak[1]%
%
\begin{equation}
	\begin{array}{@{}r@{}l@{}}
		E_c = {}
		& \{(0,O_{1,j,t_1}) \mid O_{1,j,t_1}\in V\}
		\\ {} \cup {}
		& \{(O_{i-1,j,t_{i-1}},O_{i,j,t_i}) \mid O_{i,j,t_i}\in V, i > 1\}
	\end{array}
\end{equation}
%
connect the dummy start node~$0$ to the first operation
and each operation on to its successor (if any) in the sequence for a job,
and disjunctive edges\linebreak[1]%
%
\begin{equation}
	E_d = \{(O_{i,j,t},O_{i',j',t}) \mid O_{i,j,t}\in V,O_{i',j',t}\in V, j\neq j'\}
\end{equation}
%
link operations (of distinct jobs) sharing a common tool group,
as such operations may be allocated to the same machine.

Any feasible schedule induces an acyclic subgraph $(V,E)$ of 
the disjunctive graph~$G$
such that $E_c\subseteq E$, and $(O_{i,j,t},O_{i',j',t})\in E_d\cap E$
iff $s_{i,j,t,m}+d_{i,j,t} < s_{i',j',t,m}$ for distinct jobs $j\neq j'$,
i.e., the operation
$O_{i,j,t}$ is processed before $O_{i',j',t}$ by the same machine~$m$
of tool group $t_m=t$.
Conversely,
the search for a high-quality solution can be accomplished by
determining an acyclic subgraph $(V,E)$ of~$G$ that represents a schedule
of short makespan.

For example, Table~\ref{tab:operations} shows operations
belonging to five jobs, as they can
be obtained with the parameter $n=5$ for the planning period.
Conjunctive edges connect the dummy start node~$0$ to
the operations which come first in their jobs and all operations to their successors.
In addition, mutual disjunctive edges link operations
to be processed on the same tool group.
The resulting $(N+1)\times(N+1)$ adjacency matrix, where $N$ is the
total number of operations, $0$ entries indicate the absence, and
$1$ entries the existence of edges, is given in Figure~\ref{fig:a}.

\begin{table}[ht]
	\caption{Example operations}\label{tab:operations} \centering
	\begin{tabular}{|l|l|l|l|}
		\hline
		No. & Operation & Tool group name & Avg process time (sec)\\ \hline
		$1$ & $O_{1,1,2}$ & TF\_Met\_FE\_45 & $588$ \\
		$2$ & $O_{2,1,3}$ & WE\_FE\_108 & $59$ \\
		$3$ & $O_{3,1,4}$ & WE\_FE\_83 & $62$ \\
		$4$ & $O_{4,1,5}$ & WE\_FE\_84 & $54$ \\
		$5$ & $O_{5,1,1}$ & LithoTrack\_FE\_115 & $130$ \\
		$6$ & $O_{1,2,2}$ & TF\_Met\_FE\_45    & $588$ \\
		$7$ & $O_{2,2,3}$ & WE\_FE\_108 & $59$ \\
		$8$ & $O_{3,2,4}$ & WE\_FE\_83         & $62$ \\
		$9$ & $O_{4,2,5}$ & WE\_FE\_83    & $54$ \\
		$10$ & $O_{5,2,1}$ & LithoTrack\_FE\_115 & $130$ \\
		$11$ & $O_{1,3,2}$ & TF\_Met\_FE\_45         & $588$ \\
		$12$ & $O_{2,3,3}$ & WE\_FE\_108    & $59$ \\
		$13$ & $O_{3,3,4}$ & WE\_FE\_83 & $62$ \\
		$14$ & $O_{4,3,5}$ & WE\_FE\_84         & $54$ \\
		$15$ & $O_{5,3,1}$ & LithoTrack\_FE\_115    & $130$ \\
		$16$ & $O_{1,4,2}$ & TF\_Met\_FE\_45 & $588$ \\
		$17$ & $O_{2,4,3}$ & WE\_FE\_108         & $59$ \\
		$18$ & $O_{3,4,4}$ & WE\_FE\_83    & $62$ \\
		$19$ & $O_{4,4,5}$ & WE\_FE\_84 & $54$ \\
		$20$ & $O_{5,4,1}$ & LithoTrack\_FE\_115         & $130$ \\
		$21$ & $O_{1,5,2}$ & TF\_Met\_FE\_45   & $588$ \\
		$22$ & $O_{2,5,3}$ & WE\_FE\_108 & $59$ \\
		$23$ & $O_{3,5,4}$ & WE\_FE\_83         & $62$ \\
		$24$ & $O_{4,5,5}$ & WE\_FE\_84    & $54$ \\
		$25$ & $O_{5,5,0}$ & LithoTrack\_FE\_115 & $115$ \\
		\hline
	\end{tabular}
\end{table}

\begin{table}[ht]
	\caption{Example instance}\label{tab:instance} \centering
	\begin{tabular}{|l|l|l|}
		\hline
		Lot & Product & Step \\ \hline
		$1$ & $1$ & $1$ \\
		$2$ & $2$ & $2$  \\
		$3$ & $3$ & $1$ \\
		$4$ & $4$ & $3$ \\
		$5$ & $5$ & $1$ \\
		$1$ & $1$ & $2$     \\
		$2$ & $2$ & $1$  \\
		$3$ & $3$ & $3$        \\
		$4$ & $4$ & $1$     \\
		$5$ & $5$ & $2$  \\
		\hline
	\end{tabular}
\end{table}
%

\begin{figure}[h]
	\centering
	\begin{minipage}{.45\columnwidth}
		\centering
		$\begin{array}{c@{}c}
			& \begin{array}{cccccccccc} 0 & 1 & 2 & \dots & 41 & 42 & 43 \end{array} \\
			\begin{array}{c} 0 \\ 1 \\ 2 \\ \vdots \\ 41 \\ 42 \\ 43 \end{array} &
			\left[\begin{array}{ccccccccc}
				0 & 1 & 0 & \dots & 0 & 0 & 1 \\ 
				0 & 0 & 1 & \dots & 0 & 0 & 0 \\ 
				0 & 0 & 0 & \dots & 1 & 0 & 0 \\ 
				\vdots & \vdots & \vdots & \ddots & \vdots & \vdots & \vdots \\
				0 & 0 & 0 & \dots & 0 & 1 & 0 \\ 
				0 & 0 & 0 & \dots & 0 & 0 & 1 \\ 
				0 & 0 & 1 & \dots & 0 & 0 & 0 \\
			\end{array}\right]
		\end{array}$
		\caption{Adjacency matrix for instance in Table~\ref{tab:instance}}
		\label{fig:a}
	\end{minipage}\hfill
	\begin{minipage}{.45\columnwidth}
		\centering
		$\begin{array}{c@{}c}
			& \begin{array}{cccccccccccc} 1 & 2 & 3 & \dots & 10 & 11 & 12 \end{array} \\
			\begin{array}{c} 0 \\ 1 \\ 2 \\ \vdots \\ 41 \\ 42 \\ 43 \end{array} &
			\left[\begin{array}{cccccccccccc}
				0 & 1 & 0 & \dots & 0 & 0 & 0 \\ 
				0 & 0 & 1 & \dots & 0 & 0 & 0 \\ 
				0 & 0 & 0 & \dots & 1 & 0 & 0 \\ 
				\vdots & \vdots & \vdots & \ddots & \vdots & \vdots & \vdots \\
				0 & 0 & 0 & \dots & 0 & 1 & 0 \\ 
				0 & 0 & 0 & \dots & 0 & 0 & 1 \\ 
				0 & 0 & 1 & \dots & 0 & 0 & 0 \\
			\end{array}\right]
		\end{array}$
		\caption{Machine matrix for instance in Table~\ref{tab:instance}}
		\label{fig:b}
	\end{minipage}
\end{figure}

As initial pheromone level on edges $e\in E_c\cup E_d$,
we take $\tau_y=1$ by default.
In general, representing pheromone levels by an $(N+1)\times(N+1)$
matrix similar to the adjacency matrix,
the entries~$\tau_{e}$ are initialized according to the following condition:\linebreak[1]%

\begin{equation}
	\tau_{e} =
	\begin{cases}
		\tau_y & \text{if $e\in E_c\cup E_d$} \\
		0      & \text{otherwise}
	\end{cases}
\end{equation} 

With $\tau_y=1$, this reproduces the adjacency matrix in Figure~\ref{fig:a}
as initial pheromone matrix for our example.

We additionally represent the possible machine assignments
by an $(N+1)\times M$ machine matrix, where $M$ is the total number of
machines.
For example, with two machines per tool group and the mapping
$t_m=\lceil \frac{m}{2} \rceil$ from machine identifiers
$m\in \{1,\dots,12\}$ to the tool groups $t\in\{1,\dots,6\}$,
responsible for processing the remaining operations in Table~\ref{tab:instance},
we obtain the machine matrix shown in Figure~\ref{fig:b}.

\subsection{GS Module}

The general goal of greedy search methods consists of using heuristic decisions
to find high-quality, but not necessarily optimal solutions in short time.
Within GSACO-I, each ant applies greedy search to efficiently construct some
feasible schedule for a given SMSP instance.
The respective GS procedure, outlined by the pseudo-code in Algorithm~\ref{gs-m} and Algorithm~\ref{gs-o},
includes two phases: 
operation sequencing and machine assignment.
%

\begin{algorithm}[t]
	\caption{Greedy Search (GS) Makespan}
	\label{gs-m}
	\KwOut{schedule and selected edges of an an\rlap{t}}
	$\mathit{sequence}\leftarrow[]$\;
	$\mathit{selected}\leftarrow\emptyset$\;
	$\mathit{next}\leftarrow\{(0,O_{1,j,t}) \mid j\in\{1,\dots,J\}\}$\;
	%	\lForEach{$m\in\{1,\dots,M\}$}{$f_m\leftarrow 0$}
	\While{$\mathit{next}\neq\emptyset$}{
		%	$\mathit{\tau}=\sum_{e\in\mathit{next}}\tau_e$\;
		\lForEach{$e\in\mathit{next}$}{%}
		$p_e\leftarrow \frac{\tau_e}{\sum_{e'\in\mathit{next}}\tau_{e'}}$}
	Randomly select an edge $e$ based on $p_e$\;
	\For{selected edge $e=(O',O_{i,j,t})$}{
		$\mathit{sequence}.\mathrm{enqueue}(O_{i,j,t})$\;
		$\mathit{selected}\leftarrow\mathit{selected}\cup\{e\}$\;
		$\mathit{next}\leftarrow\{(O_1,O_2)\in\mathit{next} \mid O_2\neq O_{i,j,t}\}$\;
		$\begin{array}{@{}r@{}l@{}}
			E \leftarrow {} &		
			\{(O_{i,j,t},O_{{i+1},j,t'}) \mid i<n_j\}
			%	  \cup {}
			\\ {} \cup {} &
			\{(O_{i,j,t},O_{i',j',t}) \mid (O_1,O_{i',j',t})\in\mathit{next}\}\text{\;}
		\end{array}$
		$\mathit{next}\leftarrow\mathit{next}\cup E$\;
}}
\lForEach{$m\in\{1,\dots,M\}$}{$a_m\leftarrow 0$}
\While{$\mathit{sequence}\neq[]$}{
	$O_{i,j,t}\leftarrow\mathit{sequence}.\mathrm{dequeue}()$\;
	$a \leftarrow \infty$\;
	%	$b \leftarrow 0$\;
	\ForEach{$m$ \KwFrom $1$ to \KwTo $M$}{
		\If{$t_m = t$ and $a_m < a$}{
			$a\leftarrow a_m$\;
			$b\leftarrow m$\;
		}
	}
	%	\If{$0<b$}{
		$\begin{array}{@{}r@{}l@{}}
			s_{i,j,t,b}\leftarrow 
			\max(&\{a\}
			\\ {} \cup {} &
			\{s_{i-1,j,t',m}+d_{i-1,j,t'}\mid{}i>1\})\text{\;}
		\end{array}
		$
		$a_b \leftarrow s_{i,j,t,b}+d_{i,j,t}$\;
		%	}
}
\Return $\langle\{s_{i,j,t,m} \mid (O',O_{i,j,t})\in\mathit{selected}\},{}$\rlap{$\mathit{selected}\rangle$\;}
\end{algorithm}


\begin{algorithm}[t]
\caption{Greedy Search (GS) Operations}
\label{gs-o}
\KwOut{schedule and selected edges of an an\rlap{t}}
$\mathit{sequence}\leftarrow[]$\;
$\mathit{selected}\leftarrow\emptyset$\;
$\mathit{period}\leftarrow{h}$\;
$\mathit{next}\leftarrow\{(0,O_{1,j,t}) \mid j\in\{1,\dots,J\}\}$\;
%	\lForEach{$m\in\{1,\dots,M\}$}{$f_m\leftarrow 0$}
\While{$\mathit{next}\neq\emptyset$}{
	%	$\mathit{\tau}=\sum_{e\in\mathit{next}}\tau_e$\;
	\lForEach{$e\in\mathit{next}$}{%}
	$p_e\leftarrow \frac{\tau_e}{\sum_{e'\in\mathit{next}}\tau_{e'}}$}
Randomly select an edge $e$ based on $p_e$\;
$\mathit{end time}\leftarrow{e}$\;
\For{selected edge $e=(O',O_{i,j,t})$}{
	$\mathit{e}\leftarrow{start time + process time}$\;
	\If{$\mathit{e}<{period}$}{countinue}
	$\mathit{sequence}.\mathrm{enqueue}(O_{i,j,t})$\;
	$\mathit{selected}\leftarrow\mathit{selected}\cup\{e\}$\;
	$\mathit{next}\leftarrow\{(O_1,O_2)\in\mathit{next} \mid O_2\neq O_{i,j,t}\}$\;
	$\begin{array}{@{}r@{}l@{}}
		E \leftarrow {} &		
		\{(O_{i,j,t},O_{{i+1},j,t'}) \mid i<n_j\}
		%	  \cup {}
		\\ {} \cup {} &
		\{(O_{i,j,t},O_{i',j',t}) \mid (O_1,O_{i',j',t})\in\mathit{next}\}\text{\;}
	\end{array}$
	$\mathit{next}\leftarrow\mathit{next}\cup E$\;
}}
\lForEach{$m\in\{1,\dots,M\}$}{$a_m\leftarrow 0$}
\While{$\mathit{sequence}\neq[]$}{
$O_{i,j,t}\leftarrow\mathit{sequence}.\mathrm{dequeue}()$\;
$a \leftarrow \infty$\;
%	$b \leftarrow 0$\;
\ForEach{$m$ \KwFrom $1$ to \KwTo $M$}{
	\If{$t_m = t$ and $a_m < a$}{
		$a\leftarrow a_m$\;
		$b\leftarrow m$\;
	}
}
%	\If{$0<b$}{
	$\begin{array}{@{}r@{}l@{}}
		s_{i,j,t,b}\leftarrow 
		\max(&\{a\}
		\\ {} \cup {} &
		\{s_{i-1,j,t',m}+d_{i-1,j,t'}\mid{}i>1\})\text{\;}
	\end{array}
	$
	$a_b \leftarrow s_{i,j,t,b}+d_{i,j,t}$\;
	%	}
}
\Return $\langle\{s_{i,j,t,m} \mid (O',O_{i,j,t})\in\mathit{selected}\},{}$\rlap{$\mathit{selected}\rangle$\;}
\end{algorithm}

\subsubsection{GS Makespan}
To minimize makespan, the first phase constructs an ordered list of all operations within an SMSP instance, planned to schedule. This is achieved through a probabilistic decision-making rule derived from the pheromone matrix, which selects edges $(O',O_{i,j,t})$ and sequentially adds their target operations 
$O_{i,j,t}$ to the list. To ensure the feasibility of the resulting schedule, the prerequisite operation 
$O_{i-1,j,t'}$ for the same job~$j$ must already belong to the sequence in case $i>1$. 
This requirement is met by maintaining a set  $\mathit{next}$
of selectable conjunctive and disjunctive edges $(O',O_{i,j,t})$
such that $O'$ is the dummy start node~$0$ or already in sequence,
while $O_{i,j,t}$ is the first yet unsequenced operation of its job~$j$. 

The selection of an edge leading to an unsequenced operation continues until every operation and its corresponding targeting edge has been processed. It is important to note that the chosen disjunctive edges connect operations based on their tool groups, meaning they do not dictate the machine assignments, which are determined in the subsequent phase.

With a complete sequence of operations available, the second phase involves the allocation of each operation to the earliest available machine. This machine assignment step follows the order established in the first phase to create a feasible schedule that assigns machines and start times to all operations.


\subsubsection{GS Operations}
For optimizing operations, the first phase constructs a sequence comprising of all operations
of an SMSP instance that can fit in planning period.
Similarly, a probabilistic decision rule based on the pheromone
matrix selects edges $(O',O_{i,j,t})$
and adds their target operations $O_{i,j,t}$
to the sequence one by one. The operations can only be added to the sequence if the ending time is within the period $h$.
To ensure the feasibility of a resulting schedule,
the predecessor operation $O_{i-1,j,t'}$ of the same job~$j$
must already belong to the sequence in case $i>1$.
This is accomplished by maintaining a set $\mathit{next}$
of selectable conjunctive and disjunctive edges $(O',O_{i,j,t})$
such that $O'$ is the dummy start node~$0$ or already in sequence,
while $O_{i,j,t}$ is the first yet unsequenced operation of its job~$j$.

The process of selecting some edge to a yet unsequenced operation
is repeated until each operation along with an edge targeting it
has been processed.
Note that selected disjunctive edges link operations based on tool groups,
i.e., they do not reflect a machine assignment to be made in the second phase.

With a sequence of operations at hand,
the second phase allocates operations one by one to an earliest
available machine.
The machine assignment process allocates operations
according to the sequence from the first phase, yielding a
feasible schedule that assigns the machines and start times for all operations.

The schedule for minimizing makespan is shown Figure~\ref{fig:sch-makespan}, and a schedule for 15 minutes is shown in Figure~\ref{fig:sch-operations}.

\begin{figure}[t]
	\includegraphics[width=\textwidth]{schedule_example_makespan.png}
	\caption{Feasible schedule for the instance in Table~\ref{tab:instance}}
	\label{fig:sch-makespan}
\end{figure}
\begin{figure}[ht]
	\includegraphics[width=\textwidth]{schedule_example_operations.png}
	\caption{Feasible 15-minutes schedule for the instance in Table~\ref{tab:instance}}
	\label{fig:sch-operations}
\end{figure}


\subsection{Evaluation Module}
While each ant independently constructs a feasible schedule by means of the
GS procedure, the evaluation module collects the results obtained
by the ants in a GSACO iteration.
Among them, a schedule of shortest makespan is determined as outcome of the
iteration and stored as new best solution in case it improves over the schedules found in previous iterations (if any).
After evaporating pheromones~$\tau_e$ by $\rho\cdot\tau_e$,
where $\tau_z$ remains as minimum pheromone level if the obtained value
would be smaller,
the edges~$e$ that have been selected by the GS procedure for constructing the current best schedule receive a pheromone contribution and are updated to $\tau_e+c$.

Note that our contribution parameter $c$ is a constant,
while approaches in the literature often take the inverse of an objective value
\cite{turkyilmaz2020research}, i.e., of the makespan in our case.
The latter requires careful scaling to obtain non-marginal
pheromone contributions, in particular, when makespans get as large
as for SMSP instances.
We instead opt for pheromone contributions such that the
edges selected to construct best schedules are certain to have an increased chance
of getting re-selected in forthcoming iterations.

